\documentclass[11pt]{article}

\usepackage{amsfonts}

\usepackage{fancyhdr}
\usepackage{amsmath}
\usepackage{amsthm}
\usepackage{amssymb}
\usepackage{amsrefs}
\usepackage{ulem}
\usepackage[dvips]{graphicx}
\usepackage{bm}
\usepackage{cancel}
\usepackage{color}
\usepackage{statrep}

\setlength{\headheight}{26pt}
\setlength{\oddsidemargin}{0in}
\setlength{\textwidth}{6.5in}
\setlength{\textheight}{8.5in}

\topmargin 0pt
%Forrest Shortcuts
\newtheorem{defn}{Definition}
\newtheorem{thm}{Theorem}
\newtheorem{lemma}{Lemma}
\newtheorem{pf}{Proof}
\newtheorem{sol}{Solution}
\newcommand{\R}{{\ensuremath{\mathbb R}}}
\newcommand{\J}{{\ensuremath{\mathbb J}}}
\newcommand{\Z}{{\mathbb Z}}
\newcommand{\N}{{\mathbb N}}
\newcommand{\T}{{\mathbb T}}
\newcommand{\Q}{{\mathbb Q}}
\newcommand{\st}{{\text{\ s.t.\ }}}
\newcommand{\rto}{\hookrightarrow}
\newcommand{\rtto}{\hookrightarrow\rightarrow}
\newcommand{\tto}{\to\to}
\newcommand{\C}{{\mathbb C}}
\newcommand{\ep}{\epsilon}
%CJ shortcuts
\newcommand{\thin}{\thinspace}
\newcommand{\beps}{\boldsymbol{\epsilon}}
\newcommand{\bwoc}{by way of contradiction}

%Munkres formatting?
%\renewcommand{\theenumii}{\alph{enumi}}
\renewcommand{\labelenumi}{\theenumi.}
\renewcommand{\theenumii}{\alph{enumii}}
\renewcommand{\labelenumii}{(\theenumii)}

\title{HW3}
\author{Guanlin Zhang}

\lhead{Dr Devin Koestler
 \\BIOS 830} \chead{}
\rhead{Guanlin Zhang\\Spring '17} \pagestyle{fancyplain}
%\maketitle

\begin{document}
Ch7 Question $8$.
\begin{sol}
	for part $(a)$:\vskip 2mm
	Since there is only interaction between $A$ and $B$, the model is
	\begin{align*}
		Y_{ijkt}&= \mu + \alpha_i + \beta_j + \gamma_k + (\alpha\beta)_{ij} + \epsilon_{ijket}, \text{ where }\epsilon_{ijkt} \sim N(0, \sigma^2), i.i.d.
	\end{align*}
	for part $(b)$:\vskip 2mm
	the following contrasts may be of interest:
	\begin{enumerate}
		\item effect of contrast of factor A: $\bar{\tau}_{1\cdot\cdot} -\bar{\tau}_{2\cdot\cdot}$
		\item effect of contrast of factor B: $\bar{\tau}_{\cdot 1\cdot} -\bar{\tau}_{\cdot 2\cdot}$ 
		\item effect of contrast of factor C: $\bar{\tau}_{\cdot \cdot 1} -\bar{\tau}_{\cdot\cdot 2}$
		\item effect of interaction contrast of factor A and B: $\bar{\tau}_{11\cdot} - \bar{\tau}_{12\cdot} - \bar{\tau}_{21\cdot} + \bar{\tau}_{22\cdot}$
	\end{enumerate}
	for part $(c)$:\vskip 2mm
	we have the code:
	\begin{Datastep}
		data PaperTowelStrength;
	input TreatCombo $ amount $ brand $ type $ strength order @@;
	datalines;
	 111 1 1 1 3279.0  3  111 1 1 1 4330.7 15  111 1 1 1 3843.7 16
	 112 1 1 2 3260.8 11  112 1 1 2 3134.2 20  112 1 1 2 3206.7 22
	 121 1 2 1 2889.6  5  121 1 2 1 3019.5  6  121 1 2 1 2451.5 21
	 122 1 2 2 2323.0  1  122 1 2 2 2603.6  2  122 1 2 2 2893.8 14
	 211 2 1 1 2964.5  4  211 2 1 1 4067.8 10  211 2 1 1 3327.0 18
	 212 2 1 2 3114.2 12  212 2 1 2 3009.3 13  212 2 1 2 3242.0 19
	 221 2 2 1 2883.4  9  221 2 2 1 2581.4 23  221 2 2 1 2385.9 24
	 222 2 2 2 2142.3  7  222 2 2 2 2364.9  8  222 2 2 2 2189.9 17
	;
run;

proc glm data = papertowelstrength;
	class amount brand;
	model strength = amount brand amount*brand;
run;

proc glm data = papertowelstrength;
	class amount type;
	model strength = amount type amount*type;
run;

proc glm data = papertowelstrength;
	class brand type;
	model strength = brand type brand*type;
run;
	\end{Datastep}
	So for interaction between liquid amount and brand:
	\begin{center}
		\includegraphics[width = 10cm]{HW03Ch7Q8Inter_AB.jpg}
	\end{center}
	\begin{center}
		\includegraphics[width = 10cm]{HW03Ch7Q8Inter_AC.jpg}
	\end{center}
	\begin{center}
		\includegraphics[width = 10cm]{HW03Ch7Q8Inter_BC.jpg}
	\end{center}
	It seems that all lines are near to parallel so we conclude that there is no pairwise interaction.\vskip 2mm
	For part $(d)$:\vskip 2mm
	we run the reg procedure with SAS:
	\begin{Datastep}
		proc reg data=papertowelstrength;
	model strength = amount brand type AB;
run;
	\end{Datastep}
	and we got the following model dianostic sketches:
	\begin{center}
		\includegraphics[width = 10cm]{HW03Ch7Q8fitness.jpg}
	\end{center}
	residual and studentized residuals against predicted value give supportive information for equal variance assumption. QQ plot and histogram shows supportive signal for normal assumption.\vskip 2mm
	We should really sketch residuals against input order to check independence, but I am very sorry that this time I did not manage to finish in time, same as the rest two sub problems.
\end{sol}
\end{document}
