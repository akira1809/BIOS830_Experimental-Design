\documentclass[11pt]{article}

\usepackage{amsfonts}

\usepackage{fancyhdr}
\usepackage{amsmath}
\usepackage{amsthm}
\usepackage{amssymb}
\usepackage{amsrefs}
\usepackage{ulem}
\usepackage[dvips]{graphicx}
\usepackage{bm}
\usepackage{cancel}
\usepackage{color}
\usepackage{Statrep}

\setlength{\headheight}{26pt}
\setlength{\oddsidemargin}{0in}
\setlength{\textwidth}{6.5in}
\setlength{\textheight}{8.5in}

\topmargin 0pt
%Forrest Shortcuts
\newtheorem{defn}{Definition}
\newtheorem{thm}{Theorem}
\newtheorem{lemma}{Lemma}
\newtheorem{pf}{Proof}
\newtheorem{sol}{Solution}
\newcommand{\R}{{\ensuremath{\mathbb R}}}
\newcommand{\J}{{\ensuremath{\mathbb J}}}
\newcommand{\Z}{{\mathbb Z}}
\newcommand{\N}{{\mathbb N}}
\newcommand{\T}{{\mathbb T}}
\newcommand{\Q}{{\mathbb Q}}
\newcommand{\st}{{\text{\ s.t.\ }}}
\newcommand{\rto}{\hookrightarrow}
\newcommand{\rtto}{\hookrightarrow\rightarrow}
\newcommand{\tto}{\to\to}
\newcommand{\C}{{\mathbb C}}
\newcommand{\ep}{\epsilon}
%CJ shortcuts
\newcommand{\thin}{\thinspace}
\newcommand{\beps}{\boldsymbol{\epsilon}}
\newcommand{\bwoc}{by way of contradiction}

%Munkres formatting?
%\renewcommand{\theenumii}{\alph{enumi}}
\renewcommand{\labelenumi}{\theenumi.}
\renewcommand{\theenumii}{\alph{enumii}}
\renewcommand{\labelenumii}{(\theenumii)}

\title{HW3}
\author{Guanlin Zhang}

\lhead{Dr Devin Koestler
 \\BIOS 830} \chead{}
\rhead{Guanlin Zhang\\Spring '17} \pagestyle{fancyplain}
%\maketitle

\begin{document}
Ch6 Question $8$.
\begin{sol}
	We actually used SAS to do this problem:
	\begin{Datastep}
	data Weld;
	input Gauge $ Time $ Strength @@;
	datalines;
	 G1 T1 10 G1 T1 12 G1 T2 13 G1 T2 17 G1 T3 21 G1 T3 30 
	 G1 T4 18 G1 T4 16 G1 T5 17 G1 T5 21 G2 T1 15 G2 T1 19 
	 G2 T2 14 G2 T2 12 G2 T3 30 G2 T3 38 G2 T4 15 G2 T4 11 
	 G2 T5 14 G2 T5 12 G3 T1 10 G3 T1 8  G3 T2 12 G3 T2 9  
	 G3 T3 19 G3 T3 5  G3 T4 14 G3 T4 15 G3 T5 19 G3 T5 11 
	;
run;

proc glm data = weld;
	class gauge time;
	model strength = gauge time gauge*time;
	contrast 'linear gauge contrast' gauge -1 0 1;
run;
	\end{Datastep}
	We present the output before we answer each part:
	\begin{center}
		\includegraphics[width = 10cm]{HW3Ch6Q8ANOVA.jpg}
	\end{center}
	also we have the interaction plot:
	\begin{center}
		\includegraphics[width = 10cm]{HW3Ch6Q8Inter.jpg}
	\end{center}
	So for part $(a)$:\vskip 2mm
	Since the p value for interaction is $0.0242 < 0.05$, so we reject the null hypothesis. This is interpreted as, we think there should be an interaction between gage bar setting and time of weld.\vskip 2mm
	For part $(b)$: \vskip 2mm
	Just check the interaction plot above, apparently we do not have parallel pattern here between the treatment combinations, so it boost our conclusion from part $(a)$.\vskip 2mm
	For part $(c)$: \vskip 2mm
	It is {\bf NOT} sensible to inveestigate the differences between the effects of gage bar setting because there is interaction with the welding time, based on the graph of part $(b)$.\vskip 2mm
	For part $(d)$: \vskip 2mm
	Check the output we gave at the beginning for this specific trend contrast for gauge bar setting: $(-1, 0, 1)$. The pvalue is $0.0109 < 0.05$ so we reject the null, which is interpreted as that we think the linear trend in guage bar setting is {\bf NOT} negligible. 
\end{sol}
\end{document}
